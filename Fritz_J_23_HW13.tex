\documentclass{article}
\usepackage{graphicx} % Required for inserting images

\title{PHYS 20323 - HW 13}
\author{Johané Fritz}
\date{November 2023}

\usepackage{amsmath}
\usepackage{color}
\usepackage{xcolor}
\usepackage[margin=2.5cm]{geometry}
\begin{document}

\maketitle

\section{Introduction}

\pagebreak 

\begin{center}
\textbf{{\Large PHYS 20323/60323: Fall 2023 - LaTeX Example}} \\
\end{center}

\textbf{{\large 1. The following questions refers to the Table below}} 

\vspace{6pt}

\hspace{0.7cm}{\large{Note: There may be multiple answers.}}\\

\hspace{0.7cm}\begin{tabular}{|c|c|c|c|c|c|}
\hline
Name & Mass & Luminosity & Lifetime & Temperature & Radius \\
\hline
$\eta$ Car. & 60. \(\textup{M}_\odot\) & $10^6$ \(\textup{L}_\odot\) & $8.0\times10^5$ years & & \\
\hline
$\varepsilon$ Eri. & 6.0 \(\textup{M}_\odot\) & $10^3$ \(\textup{L}_\odot\) & & 20,000 K & \\
\hline
$\delta$ Scu. & 2.0  \(\textup{M}_\odot\) & & $5.0\times10^8$ years & & 2 \(\textup{R}_\odot\) \\
\hline
$\beta$ Cyg. & 1.3 \(\textup{M}_\odot\) & 3.5 \(\textup{L}_\odot\) & & & \\
\hline
$\alpha$ Cen. & 1.0 \(\textup{M}_\odot\) & & & & 1 \(\textup{R}_\odot\) \\
\hline
$\gamma$ Del. & 0.7 \(\textup{M}_\odot\) & & $4.5\times10^10$ years & 5000 K & \\
\hline 
\end{tabular} \\

\vspace{10pt}

\hspace{0.7cm} (a) (4 points) Which of these stars will produce a planetary nebula. \\

\vspace{8pt}

\hspace{0.7cm} (b) (4 points) Elements heavier than Carbon will be produced in which stars. \\

\vspace{12pt}

2. An electron is found to be in the spin state (in the $z$-basis): $x$ = A {\LARGE \(\binom{3i}{4}\)} \\

\vspace{6pt}

\hspace{0.7cm} (a) (5 points) Determine the possible values of A such that the state is normalized. \\

\vspace{12pt}

\hspace{0.7cm} (b) (5 points) Find the expectation values of the operators {\color{red} $S_{x}$}, {\color{purple} $S_{y}$}, {\color{orange} $S_{z}$} and $\Vec{S^2}$. \\

\vspace{12pt}

\hspace{0.5cm} The matrix representation in the $z$-basis for the components of electron spin operators are given by: 

\vspace{12pt}

\hspace{0.7cm}{\color{red} $S_{x} = $$\frac{\overline{\mbox{h}}}{2}$ 
 $\begin{pmatrix}
     0 & 1 \\
     1 & 0 
 \end{pmatrix}$} ; {\color{purple} \hspace{0.9cm} $S_{y} = $$\frac{\overline{\mbox{h}}}{2}$ 
 $\begin{pmatrix}
     0 & -i \\
     i & 0 
 \end{pmatrix}$} ; {\color{orange} \hspace{0.9cm} $S_{z} = $$\frac{\overline{\mbox{h}}}{2}$ 
 $\begin{pmatrix}
     1 & 0 \\
     0 & -1 
 \end{pmatrix}$} \\

\vspace{8pt}

3. The average electrostatic field in the earth's atmosphere in fair weather is approximately given:

\vspace{10pt}

\hspace{5.4cm}$\vec{E} = E_0 (Ae^{-{\alpha}z} + Be^{-{\beta}z})\hat{z}$, \hspace{5cm}(1)
 
\vspace{12pt}

\hspace{0.5cm} where A, B, $\alpha$, $\beta$, are positive constants and $z$ is the height above the (locally flat) earth surface. 

\vspace{12pt}

\hspace{0.7cm} (a) (5 points) Find the average charge density in the atmosphere as a function of height. \\

\vspace{12pt}

\hspace{0.7cm} (b) (5 points) Find the electric potential as a function height above the earth. 

\end{document}
